\section{Integral Calculus}
Remember to always include the constant of integration \(+\ C\) when evaluating indefinite integrals.
\begin{align*}
    \int x^n \, dx      &= \frac{x^{n+1}}{n+1} \ (n \neq -1 )&
    \int e^x \, dx      &= e^x &
    \int \frac{1}{x} \, dx &= \ln |x| &
    \int \ln x \, dx    &= x \ln x - x &
    \int b^x \, dx      &= \frac{b^x}{\ln b}
\end{align*}

\subsection{Integration by Substitution}
\begin{align*}
    \int f(g(x))g'(x)\, dx &= \int f(u)\, du, \quad \text{where } u = g(x)
\end{align*}

\subsection{Integration by Parts}
For choice of \( u \): use the mnemonic LIATE: Logarithmic, Inverse-trigonometric, Algebraic, Trigonometric, Exponential.
\begin{align*}
    \int u \, dv &= uv - \int v \, du & \int f(x) g'(x) \, dx &= f(x) g(x) - \int f'(x) g(x) \, dx
\end{align*}

% \subsection{Integrals of Logarithmic and Exponential Functions}
% \begin{align*}
%     \int e^x \, dx      &= e^x + C &
%     \int \ln x \, dx    &= x \ln x - x + C &
%     \int b^x \, dx      &= \frac{1}{\ln b} b^x + C
% \end{align*}

\subsection{Integrals of Trigonometric Functions (and Hyperbolic Functions)}
\begin{align*}
    % Integrals of "ordinary" trigonometric functions
    &\int \sin x \, dx = -\cos x + C &&
    \int \cos x \, dx = \sin x + C &&
    \int \sec^2 x \, dx = \tan x + C \\
    % Slightly more obscure ones
    &\int \csc^2 x \, dx = -\cot x + C &&
    \int \sec x \tan x \, dx = \sec x + C &&
    \int \csc x \cot x \, dx = -\csc x + C \\
    % Integrals that yield functions of natural logs, hyperbolic functions
    &\int \sec x \, dx = \ln |\sec x + \tan x | + C &&
    \int \tan x \, dx = \ln |\sec x | + C  &&
    \int \sinh x \, dx = \cosh x + C \\
    &\int \csc x \, dx = \ln |\csc x - \cot x | + C &&
    \int \cot x \, dx = \ln |\sin x | + C &&
    \int \cosh x \, dx = \sinh x + C
\end{align*}

\subsection{Integrals that yield Inverse Trigonometric Functions}
\begin{align*}
    \int \frac{dx}{\sqrt{a^2-x^2}} &= \sin^{-1} \left( \frac{x}{a} \right) + C & 
    \int \frac{dx}{a^2+x^2} &= \frac{1}{a} \tan^{-1} \left( \frac{x}{a} \right) + C &
    \int \frac{dx}{x \sqrt{x^2-a^2}} &= \frac{1}{a} \sec^{-1} \left| \frac{x}{a} \right| + C
\end{align*}

\subsection{Reduction Formulas for $\sin$ and $\cos$}
\begin{align*}
    \int \sin^n x \, dx &= -\frac{1}{n} \sin^{n-1} x \cos x + \frac{n-1}{n} \int \sin^{n-2} x \, dx &
    \int \cos^n x \, dx &= +\frac{1}{n} \cos^{n-1} x \sin x + \frac{n-1}{n} \int \cos^{n-2} x \, dx
\end{align*}

\subsection{Half-Angle Identities}
\begin{align*}
    \cos^2 x &= \frac{1}{2} (1 + \cos 2x) &
    \sin^2 x &= \frac{1}{2} (1 - \cos 2x) &
    \sin x \cos x &= \frac{1}{2} \sin 2x
\end{align*}

\subsection{Trigonometric Identities for Evaluating Integrals of Products of Trigonometric Powers}

\begin{align*}
    % Labels
    & \cos^n \textbf{ is odd, use } u=\sin x &&
    \sin^n \ \textbf{is odd, use }   u = \cos x &&
    \sec^n \ \textbf{is even, use }  u = \tan x &&
    \tan^n \ \textbf{is odd, use }   u = \sec x \\
    % Substitution identities
    & \cos^2 x = (1 - \sin^2 x) &&
    \sin^2 x = (1 - \cos^2 x) &&
    \sec^2 x = (1 + \tan^2 x) &&
    \tan^2 x = (\sec^2 x - 1)
\end{align*}


\subsection{Product Identities for Trigonometric Products of $\sin$, $\cos$}
\begin{align*}
\sin A \cos B &= \frac{\sin(A-B) + \sin(A+B)}{2} &
\cos A \cos B &= \frac{\cos(A-B) + \cos(A+B)}{2} &
\sin A \sin B &= \frac{\cos(A-B) - \cos(A+B)}{2}
\end{align*}

\subsection{Trigonometric Substitutions}
\begin{align*}
    \textbf{Expression} && \textbf{Substitution} && \textbf{Domain} && \textbf{Identity} \\
    \int \sqrt{a^2-x^2} \, dx &&  x = a \sin \theta, && -\frac{\pi}{2} \leq \theta \leq \frac{\pi}{2} && 1 - \sin^2 \theta = \cos^2 \theta \\
    \int \sqrt{a^2+x^2} \, dx &&  x = a \tan \theta, && -\frac{\pi}{2} < \theta < \frac{\pi}{2} && 1 + \tan^2 \theta = \sec^2 \theta \\
    \int \sqrt{x^2-a^2} \, dx &&  x = a \sec \theta, && 0 \leq \theta < \frac{\pi}{2} && \sec^2 \theta - 1 = \tan^2 \theta 
\end{align*}